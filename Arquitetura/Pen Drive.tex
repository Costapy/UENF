\documentclass[a4paper, 12pt]{article}
\usepackage[brazil]{babel}
\usepackage{graphicx}
\usepackage{url}
\usepackage{here}
\usepackage[alf]{abntex2cite}

%opening
\title{\textbf{Pen Drive}}
\author{Gabriel Costa Fassarella, Enzo Marques Souza, \\ Binha Ferraz Dauma, João Pedro Oliveira  \\ \\ UENF - CCT - LCMAT - C. Computação \\}
\date{\today}

\begin{document}
	\maketitle
	
	\newpage
	
	\section{Introdução}
	
	O pendrive, também conhecido como memória flash USB, é um dispositivo de armazenamento portátil amplamente utilizado em todo o mundo. Sua popularidade cresceu à medida que a tecnologia avançou, tornando-se uma alternativa mais prática e acessível para o armazenamento e transporte de dados. Neste seminário, vamos abordar vários aspectos importantes sobre os pendrives, incluindo seu funcionamento, tipos de memória, velocidade e capacidade de armazenamento, bem como as versões atuais e antigas e o valor de mercado.
	
	\section{Arquitetura}
	
	A arquitetura de um pen drive é geralmente simples, possuindo 3 componentes principais: a memória flash, o controlador USB e um conector USB.
	A unidade de memória flash é o componente responsável por armazenar os dados no pen drive. A memória flash costuma ser organizada em blocos de dados que podem ser tanto lidos quanto gravados pelo controlador USB.
	O controlador USB funciona como um "cerébro" do pen drive, gerenciando a relação de comunicação entre o pen drive e o dispositivo conectado (o computador por exemplo). O controlador USB faz isso traduzindo as informações entre o formato de dados do pen drive com o formato de dados que o dispositivo conectado é capaz de suportar. Além disso, esse componente tem a função de gerenciar a transferência de dados entre a memória e o dispositivo conectado. 
	E como último componente principal, existe o conector USB que permite a conexão do pen drive com o dispositivo desejado. Quando ocorre a junção entre o conector e uma porta USB, o controlador USB identifica a interação iniciando assim a comunicação entre ambos os dispositivos.
	
	\section{Tipos de Memória}
	
	Os pendrives são compostos por uma placa de circuito impresso, onde é armazenada a memória flash, um tipo de memória não volátil que permite a retenção de informações mesmo quando não há energia elétrica. Existem dois tipos principais de memória flash usados em pendrives: NAND e NOR. A memória NAND é mais comum em pendrives devido à sua alta capacidade de armazenamento, enquanto a memória NOR é usada principalmente em dispositivos de baixa capacidade.
	
	\section{Versões Atuais e Antigas}
	
	Os pendrives evoluíram significativamente desde que foram introduzidos pela primeira vez no mercado. A primeira geração de pendrives, conhecida como USB 1.0, foi lançada em 1996, mas tinha uma velocidade de transferência de dados muito lenta, de apenas 12 Mbps. A segunda geração, USB 2.0, lançada em 2000, tinha uma velocidade de transferência de dados de até 480 Mbps. A terceira geração, USB 3.0, lançada em 2008, tinha uma velocidade de transferência de dados de até 5 Gbps. Atualmente, a versão mais recente é o USB 3.2, com velocidades de transferência de dados de até 20 Gbps.
	
	\section{Velocidade e Armazenamento}
	
	A velocidade de um pendrive pode variar dependendo de sua versão e da qualidade da memória flash. Os pendrives mais recentes tendem a ter velocidades de leitura e gravação mais rápidas, enquanto os pendrives mais antigos podem ser significativamente mais lentos. O armazenamento disponível em um pendrive também varia amplamente, com tamanhos de capacidade variando de alguns megabytes até centenas de gigabytes.
	
	\section{Porta USB e Funcionamento}
	
	Antes de falarmos sobre o funcionamento dos pendrives, é necessário falar sobre as portas USB. Ela é um tipo de conexão que permite a transferência de dados entre dispositivos eletrônicos. Existem diferentes versões da porta USB, sendo as mais comuns a USB 1.0, 2.0, 3.0 e a 3.1. Cada nova versão traz melhorias em relação à velocidade de transferência de dados e capacidade de energia. Por exemplo, a USB 3.1 é capaz de transferir dados a uma velocidade de até 10 Gbps e fornecer até 100W de energia. As portas USB são amplamente utilizadas em todo o mundo, e é muito provável que você as use regularmente para carregar seu smartphone ou transferir arquivos entre dispositivos. Voltando aos pendrives, o pendrive é conectado ao computador por meio de uma dessas tais portas USB e é automaticamente reconhecido pelo sistema operacional do computador. Uma vez conectado, o pendrive pode serv usado para armazenar e transferir arquivos entre o computador e o dispositivo.
	
	\section{Valor de Mercado}
	
	O valor de mercado de um pendrive pode variar amplamente, dependendo de sua capacidade de armazenamento e velocidade. Os pendrives com maior capacidade de armazenamento e velocidade geralmente são mais caros do que os modelos mais antigos ou mais lentos. No entanto, devido à grande concorrência no mercado de pendrives, é possível encontrar opções acessíveis que ainda oferecem bom desempenho.
	
	\section{Conclusão}
	
	Os pendrives são uma tecnologia importante e conveniente para o armazenamento e transporte de dados. Com várias opções de capacidade de armazenamento e velocidade, eles podem ser uma ferramenta útil para qualquer pessoa que precise transportar arquivos.
	
	\newpage
	\section{Referências Bibliográficas}
	
\begin{itemize}
	\item \textit{USB Flash Drives for Personal Use}, Kingston Technology. \url{https://www.kingston.com/en/usb-flash-drives/personal?sortby=nameatz&use=personal%20storage}
	
	\item \textit{USB: Port Types and Speeds Compared}, Tripp Lite. \url{https://tripplite.eaton.com/products/usb-connectivity-types-standards}
	
	\item \textit{USB Flash Drive Speed Tests}, USB FlashSpeed. \url{https://usbspeed.nirsoft.net/}
	
	\item \textit{The Best USB Flash Drives for 2022}, PCMag. \url{https://www.pcmag.com/picks/top-rated-usb-flash-drives}
	
	\item \textit{How Solid-state Drives Work}, HowStuffWorks. \url{https://computer.howstuffworks.com/solid-state-drive.htm}
	
	\item \textit{USB Flash Drive Market Global Industry Assessment \& Forecast}, Vantage Market Research. \url{https://www.vantagemarketresearch.com/industry-report/usb-flash-drive-market-1465}
	
	\item \textit{Document Library}, USB-IF (USB Implementers Forum). \url{https://www.usb.org/documents}
	
	\item \textit{How USB Ports Work}, HowStuffWorks. \url{https://computer.howstuffworks.com/usb.htm#:~:text=The%20rectangular%20socket%20is%20a,asks%20for%20the%20driver%20disk.}
\end{itemize}
	
\end{document}