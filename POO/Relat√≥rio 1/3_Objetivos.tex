%Os objetivos do experimento devem ser definidos de forma clara e sucinta (1 a 3 linhas).

-Photogate timer with memory.

-PASCO scientific ME-9215A.

-Trilho de ar - PASCO scientific.

-Carrinho com encaixe para trilho de ar.

-Triple beam balance. 

-Pesos auxiliares

-Objetos de ressalto para trilho de ar.\\

Para realizarmos o experimento de modo que ele seja do tipo uniformemente variado, usaremos a gravidade para ser esse fator de aceleração. Com os objetos de ressalto vamos causar uma inclinação no trilho de ar, decompondo o vetor de aceleração gravitacional de acordo com o ângulo entre a superfície e o trilho de ar. 
Além disso, devemos posicionar o Photogate para a medição de tempo determinando seu respectivo erro, junto a medição na régua anexa ao trilho de ar, pelo diâmetro da haste que sustenta o aparelho de detecção. 
Com essa estrutura já montada, devemos pesar o carrinho, sem os pesos auxiliares, com a balança mencionada entre os materiais utilizados para o experimento e aí então realizarmos o procedimento, que consiste em encaixar o carrinho no trilho de ar e deixar com que a gravidade possa conduzi lo da sua posição inicial até a posição final, indicada na tabela preenchida em laboratório calculando assim, com auxílio do photogate, o tempo que o carrinho demora para passar pelo primeiro instrumento e chegar no segundo.
O processo mencionado no parágrafo acima, é idêntico para o carrinho anexado dos pesos auxiliares. Já devemos anexar os pesos auxiliares ao carrinho encaixando o espaço vazado na estrutura do peso pela estrutura cilíndrica soldada no carrinho. Desse modo realizar o mesmo procedimento, que foi feito para o carrinho sem peso, agora para o carrinho com peso.

