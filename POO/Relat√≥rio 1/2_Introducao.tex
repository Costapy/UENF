%Insira sua introdução aqui (3 parágrafos é suficiente). Apague e insira seu texto.

O movimento retilíneo uniformemente variado se dá quando um corpo ou alguma partícula tem energia cinética cuja velocidade varia com o tempo, tendo a presença de uma aceleração, ou seja. No caso da Terra, há uma aceleração gravitacional de aproximadamente 9,8 m/s, no qual os corpos são atraídos para o centro dela através de uma força denominada peso. 

Isto posto, através desse experimento, foi possível observar a presença da aceleração, descrevendo o movimento retilíneo uniformemente variado e também a veracidade da fórmula horária do MRUV.

Sendo assim, o experimento teve como objetivo observar o movimento retilíneo uniformemente variado através de um carrinho com diferentes massas e um trilho de ar, que torna o atrito quase nulo, e observar que a massa não influencia na posição em função do tempo. Portanto, o tempo médio de cada carrinho com massas diferentes deve dar aproximadamente igual em um mesmo intervalo de tempo.

