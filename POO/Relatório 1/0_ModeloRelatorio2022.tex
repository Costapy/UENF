\documentclass[a4paper,openany,oneside]{abntex2}
\usepackage[T1]{fontenc}
\usepackage{graphicx}
\usepackage{amsmath}
\usepackage{float}
\usepackage{multicol,multirow}
\usepackage[bottom=2.5cm,top=2.5cm,left=2.0cm,right=2.0cm]{geometry}
\usepackage{indentfirst}
\usepackage{hyperref}
\hypersetup{colorlinks,citecolor=black,filecolor=blacke,linkcolor=blue,urlcolor=blue} %%%%
% cleardoublepage se este documento nao for article
% ---

\renewcommand\thesection{\arabic{section}} % não inclui capítulo na mumeração da seção
\setcounter{page}{0} % não conta a numeração da capa
\pagestyle{myheadings} %numerar página sem títulos extras
\usepackage[abnt-emphasize=bf,alf,bibjustif,versalete]{abntex2cite}
\usepackage{lipsum} % pacote para escrever texto aleatório.
%Não apague os comandos acima. Abaixo você pode colocar os pacotes que necessitar usar.






\begin{document}
\title{Template dos relatórios de Física da UENF}

%%%%% Verifique se o professor aceita arquivo em pdf e se a formatação é a exigida. No LCFIS estamos buscando uniformizar o modelo para facilitar o aprendizado dos alunos. 


\begin{center}
%\hspace{-0.5cm}
\raisebox{-2\baselineskip}{\includegraphics[width=0.15\textwidth]
{Figuras/logouenf.png}}
\hfill
\begin{minipage}[b]{0.80\textwidth}
  \textbf{\Large Universidade Estadual do Norte Fluminense Darcy Ribeiro}\\
  \centering
    \textbf{\Large Centro de Ciência e Tecnologia}\\
  \textbf{\large Laboratório de Física Geral I}
\end{minipage}
\vspace{180pt}

        \huge{\textbf{Relatório N$^\circ$ 1}}\\ %Entre aqui com o número do relatório
        \Large{Movimento Retilíneo Uniformemente Variado}\\ %Entre com o título do experimento
        
        \vspace{100pt}
        
        \hfill Turma E 
        
        \vspace{10pt}
        
        \hfill Ciência da Computação
        
        \vspace{30pt} %Substitua os nomes e matrículas fictícios pelos nomes dos participantes do grupo com os seus respectivos números de matrícula.

    
        \hfill Felipe Garcia \hspace{20pt} Matrícula: 20221100007  \\
        \hfill Pablo Chaves  \hspace{20pt} Matrícula: 20221100016  \\
        \hfill Matheus Gama   \hspace{20pt} Matrícula: 20221100021\\

        \vspace{30pt}
        \hfill {Professor: Marcelo Massunaga}\\ %Entre com o nome do professor
        
     
        \vspace{\fill}
        \LARGE \textbf{\today}
          
	\end{center}



%% o índice remissivo pode ser retirado se excluir o texto desde esse comentário até a próxima sequência de %, não se esqueça também de excluir os hyperref lá no alto, estarão marcados com símbolos.
\newpage

%\Large\tableofcontents
%\thispagestyle{empty}
             %%%%%
%\newpage


\mainmatter

\section{Introdução}
%Insira sua introdução aqui (3 parágrafos é suficiente). Apague e insira seu texto.

O movimento retilíneo uniformemente variado se dá quando um corpo ou alguma partícula tem energia cinética cuja velocidade varia com o tempo, tendo a presença de uma aceleração, ou seja. No caso da Terra, há uma aceleração gravitacional de aproximadamente 9,8 m/s, no qual os corpos são atraídos para o centro dela através de uma força denominada peso. 

Isto posto, através desse experimento, foi possível observar a presença da aceleração, descrevendo o movimento retilíneo uniformemente variado e também a veracidade da fórmula horária do MRUV.

Sendo assim, o experimento teve como objetivo observar o movimento retilíneo uniformemente variado através de um carrinho com diferentes massas e um trilho de ar, que torna o atrito quase nulo, e observar que a massa não influencia na posição em função do tempo. Portanto, o tempo médio de cada carrinho com massas diferentes deve dar aproximadamente igual em um mesmo intervalo de tempo.



\subsection{Fundamentação teórica}
%Coloque a fundamentação teoria aqui. Coloque as principais equações. Veja o exemplo  %exemplo de como colocar duas citações ou mais.

%Veja o ambiente que a equação está colocada. 
% Apague e coloque o seu texto.
Para realizarmos esse experimento foi de grande valia a fundamentação teórica obtida na disciplina de Física 1, de posse dessas informações conseguimos identificar melhor o tipo do movimento e as forças que atuam no corpo em questão além de medirmos o tempo para que o evento se concretizasse. 

Como esse experimento requer uma aceleração constante, o ideal é usarmos a natureza a nosso favor e utilizarmos uma constante física para fornecer essa aceleração constante que precisamos, nada melhor do que a gravidade para nos auxiliar nesse experimento, entretanto a gravidade atrai tudo para o centro da terra, todavia nosso material utilizado para esse experimento é de características retilíneas, portanto devemos utilizar apenas uma porcentagem da gravidade de modo que possamos conciliar o nosso instrumento com a nossa grandeza constante. Nesse instante deve entrar em ação o nosso conhecimento sobre vetores, ora se o vetor gravidade atrai tudo para o centro da terra, na direção y, o melhor caminho é decompor esse vetor utilizando um ângulo entre a superfície do trilho de ar e o próprio trilho de ar. Decompondo esse vetor gravidade teremos que a componente x desse vetor vale o módulo do vetor gravidade multiplicado pelo cosseno do ângulo entre a superfície e o trilho de ar. Logo esse será o nosso valor referência para aceleração, tendo em vista que a gravidade é constante e que esse ângulo também é constante.

Fórmulas usadas na construção do relatório:

\begin{equation}\label{eq1}
\sigma = \sqrt{\frac{\sum_{i=1}^n (t_i-\overline{t})^2}{(n-1)}}
\end{equation}

\begin{equation}\label{eq2}
\Delta \overline{t}=\frac{\sigma}{\sqrt n}
\end{equation}

\begin{equation}\label{eq2}
x=x_0+v_0t+\frac{at^2}{2}
\end{equation}


\section{Procedimento Experimental}
%Os objetivos do experimento devem ser definidos de forma clara e sucinta (1 a 3 linhas).

-Photogate timer with memory.

-PASCO scientific ME-9215A.

-Trilho de ar - PASCO scientific.

-Carrinho com encaixe para trilho de ar.

-Triple beam balance. 

-Pesos auxiliares

-Objetos de ressalto para trilho de ar.\\

Para realizarmos o experimento de modo que ele seja do tipo uniformemente variado, usaremos a gravidade para ser esse fator de aceleração. Com os objetos de ressalto vamos causar uma inclinação no trilho de ar, decompondo o vetor de aceleração gravitacional de acordo com o ângulo entre a superfície e o trilho de ar. 
Além disso, devemos posicionar o Photogate para a medição de tempo determinando seu respectivo erro, junto a medição na régua anexa ao trilho de ar, pelo diâmetro da haste que sustenta o aparelho de detecção. 
Com essa estrutura já montada, devemos pesar o carrinho, sem os pesos auxiliares, com a balança mencionada entre os materiais utilizados para o experimento e aí então realizarmos o procedimento, que consiste em encaixar o carrinho no trilho de ar e deixar com que a gravidade possa conduzi lo da sua posição inicial até a posição final, indicada na tabela preenchida em laboratório calculando assim, com auxílio do photogate, o tempo que o carrinho demora para passar pelo primeiro instrumento e chegar no segundo.
O processo mencionado no parágrafo acima, é idêntico para o carrinho anexado dos pesos auxiliares. Já devemos anexar os pesos auxiliares ao carrinho encaixando o espaço vazado na estrutura do peso pela estrutura cilíndrica soldada no carrinho. Desse modo realizar o mesmo procedimento, que foi feito para o carrinho sem peso, agora para o carrinho com peso.



\section{Resultados}
% Coloque os resultados e faça a análise. Não será aceito somente figuras ou tabelas sem discussão.

%Dicas:
% ! obriga a figura ficar nessa posição
% h = here (aqui), t = top (topo), b = bottom (parte de baixo) e p = page (página) 
% Figuras, equações e tabelas devem ter esse comando para que você controle as suas posições no texto. 


%*\begin{figure}[H]
 %   \centering
 %   \caption{Foto do Picachu acenando.}
 %   \includegraphics[scale=0.7]{Figuras/picachu.jpg}
%    \legend{Fonte: deve citar a fonte \cite{livro4}.}
%    \label{fig1}
%\end{figure}

%\lipsum[2] 

%Aprenda a fazer tabelas no \LaTeX.
%\url{https://www.learnlatex.org/pt/lesson-08}

\begin{table}[h!]
\caption{Posição e erro do fotogate com display}
    \centering
    \begin{tabular}{c|c}
  \toprule
  \multicolumn{1}{c}{x_0 (cm)} & \multicolumn{1}{c}{\Delta x (cm)} \\
  \midrule
  20,65 & 0,60\\
  \bottomrule
\end{tabular}
\label{tabela}
\legend{Fonte: Procedimento realizado em laboratório pelos integrantes}
\end{table}

\begin{table}[h!]
\caption{Resultados das medidas para Massa = 179,45g}
    \centering
    \begin{tabular}{llllllll}
  \toprule
  x (cm) & t_1 (s) & t_2 (s) & t_3 (s) & t_4 (s) & t_5 (s) & \overline{t} (s) & \Delta \overline{t} (s)\\
  \midrule
  40,00 & 1,617 & 1,613 & 1,614 & 1,616 & 1,641 & 1,620 & 5,248 \times 10^-3\\
  50,00 & 1,986 & 1,988 & 1,988 & 1,986 & 1,990 & 1,988 & 7,483 \times 10^-4\\
  60,00 & 2,301 & 2,324 & 2,310 & 2,309 & 2,301 & 2,319 & 3,957 \times 10^-3\\
  70,00 & 2,595 & 2,630 & 2,609 & 2,594 & 2,592 & 2,604 & 7,162 \times 10^-3\\
  80,00 & 2,841 & 2,845 & 2,836 & 2,843 & 2,842 & 2,841 & 1,503 \times 10^-3\\
  90,00 & 3,063 & 3,065 & 3,069 & 3,086 & 3,080 & 3,073 & 4,456 \times 10^-3\\
  100,00 & 3,298 & 3,298 & 3,295 & 3,296 & 3,306 & 3,299 & 1,939 \times 10^-3\\
  110,00 & 3,502 & 3,506 & 3,497 & 3,499 & 3,503 & 3,501 & 1,568 \times 10^-3\\
  120,00 & 3,689 & 3,697 & 3,696 & 3,707 & 3,703 & 3,698 & 3,092 \times 10^-3\\
  130,00 & 3,881 & 3,900 & 3,846 & 3,879 & 1,896 & 3,880 & 9,522 \times 10^-3\\
  \bottomrule
\end{tabular}
\label{tabela}
\legend{Fonte: Procedimento realizado em laboratório pelos integrantes}
\end{table}

\begin{table}[h!]
\caption{Resultados das medidas para Massa = 279,50g}
    \centering
    \begin{tabular}{llllllll}
  \toprule
  x (cm) & t_1 (s) & t_2 (s) & t_3 (s) & t_4 (s) & t_5 (s) & \overline{t} (s) & \Delta \overline{t} (s)\\
  \midrule
  40,00 & 1,611 & 1,613 & 1,609 & 1,629 & 1,617 & 1,616 & 5,555 \times 10^-3\\
  50,00 & 1,992 & 1,987 & 1,988 & 2,016 & 1,986 & 1,994 & 5,643 \times 10^-3\\
  60,00 & 2,301 & 2,298 & 2,316 & 2,298 & 2,299 & 2,302 & 3,444 \times 10^-3\\
  70,00 & 2,600 & 2,611 & 2,588 & 2,584 & 2,585 & 2,594 & 5,202 \times 10^-3\\
  80,00 & 2,847 & 2,850 & 2,835 & 2,839 & 2,862 & 2,847 & 4,697 \times 10^-3\\
  90,00 & 3,070 & 3,071 & 3,068 & 3,071 & 3,081 & 3,072 & 2,267 \times 10^-3\\
  100,00 & 3,295 & 3,311 & 3,306 & 3,306 & 3,292 & 3,302 & 3,619 \times 10^-3\\
  110,00 & 3,493 & 3,498 & 3,498 & 3,506 & 3,498 & 3,499 & 2,088 \times 10^-3\\
  120,00 & 3,695 & 3,693 & 3,700 & 3,695 & 3,696 & 3,696 & 1,158 \times 10^-3\\
  130,00 & 3,898 & 3,881 & 3,882 & 3,875 & 1,868 & 3,881 & 4,974 \times 10^-3\\
  \bottomrule
\end{tabular}
\label{tabela}
\legend{Fonte: Procedimento realizado em laboratório pelos integrantes}
\end{table}


\subsection{Análise dos Resultados e Discussão}
O experimento realizado demonstrou o que as fórmulas do movimento, obtidas da física clássica, já apontavam. No primeiro momento a medição do tempo, sem o peso, é bem próxima ao segundo, com os pesos auxiliares. A força resultante atuando no carrinho está apontando para a mesma direção da componente em x do vetor gravidade (considere o eixo das abscissas paralelo à superfície do trilho de ar), e tem seu módulo obtido pela multiplicação da massa pela componente x do vetor gravidade. Como não há movimento na direção y, podemos dizer que o peso do carrinho se equipara à força normal exercida pelo trilho de ar sobre o carrinho. Com essas informações é possível identificar que, entre o primeiro momento e o segundo, a aceleração não muda. Com isso as outras variáveis de comparação permanecem constantes tais quais, posição final; posição inicial; velocidade inicial; velocidade final e aceleração, portanto na faixa em que a posição final é 1 metro, temos que a diferença entre seus respectivos tempos médio está na casa dos milésimos de segundo, sendo contemplado com a variação do tempo médio com isso, dentro da “Margem de Erro”. E com a simples observação da tabela 3 é possível identificar esse mesmo comportamento nas outras faixas de medição.


\begin{table}[h!]
\caption{Resultados das medidas do MRUV}
    \centering
    \begin{tabular}{lll}
  \toprule
          Massa = 179,45 g\\
  x (cm) & \overline{t} (s) & \overline{t}^2 (s^2)\\
  \midrule
  40,00 & 1,620 & 2,624\\
  50,00 & 1,988 & 3,952\\
  60,00 & 2,319 & 5,378\\
  70,00 & 2,604 & 6,781\\
  80,00 & 2,841 & 8,071\\
  90,00 & 3,073 & 9,443\\
  100,00 & 3,299 & 10,883\\
  110,00 & 3,501 & 12,257\\
  120,00 & 3,698 & 13,675\\
  130,00 & 3,880 & 15,054\\
  \bottomrule
\end{tabular}
\label{tabela}
\legend{Fonte: Procedimento realizado em laboratório pelos integrantes}
\end{table}

\begin{table}[h!]
\caption{Resultados das medidas do MRUV}
    \centering
    \begin{tabular}{lll}
  \toprule
          Massa = 279,50 g\\
  x (cm) & \overline{t} (s) & \overline{t}^2 (s^2)\\
  \midrule
  40,00 & 1,616 & 2,611\\
  50,00 & 1,994 & 3,975\\
  60,00 & 2,302 & 5,301\\
  70,00 & 2,594 & 6,727\\
  80,00 & 2,847 & 8,103\\
  90,00 & 3,072 & 9,438\\
  100,00 & 3,302 & 10,903\\
  110,00 & 3,499 & 12,240\\
  120,00 & 3,696 & 13,659\\
  130,00 & 3,881 & 15,061\\
  \bottomrule
\end{tabular}
\label{tabela}
\legend{Fonte: Procedimento realizado em laboratório pelos integrantes}
\end{table}

\section{Conclusão}
%Apague e coloque aqui a conclusão dos resultados.
De acordo com os resultados obtidos pelo experimento, foi possível observar que a massa não influencia no movimento retilíneo uniformemente variado, como previsto anteriormente, visto que os resultados de ambos carrinhos foram praticamente iguais. Além disso, de acordo com o gráfico feito, da posição em função do tempo ao quadrado, é possível notar que a aceleração é constante, visto que o resultado deu uma reta que passa bem próximo de todos os pontos, e como a massa não influencia na aceleração, ambos resultados dos experimentos deram muito similares, por isso ambas retas quase se coincidem.



\bibliography{8_referencias}
\begin{itemize}
    \item HALLIDAY, D. e RESNICK, R., Fundamentos da Física, 8 ed., vol. 1
    \item HALLIDAY, D. e RESNICK, R., Fundamentos da Física, 8 ed., vol. 2
\end{itemize}

\end{document}




