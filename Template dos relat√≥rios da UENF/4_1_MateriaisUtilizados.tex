O experimento realizado demonstrou o que as fórmulas do movimento, obtidas da física clássica, já apontavam. No primeiro momento a medição do tempo, sem o peso, é bem próxima ao segundo, com os pesos auxiliares. A força resultante atuando no carrinho está apontando para a mesma direção da componente em x do vetor gravidade (considere o eixo das abscissas paralelo à superfície do trilho de ar), e tem seu módulo obtido pela multiplicação da massa pela componente x do vetor gravidade. Como não há movimento na direção y, podemos dizer que o peso do carrinho se equipara à força normal exercida pelo trilho de ar sobre o carrinho. Com essas informações é possível identificar que, entre o primeiro momento e o segundo, a aceleração não muda. Com isso as outras variáveis de comparação permanecem constantes tais quais, posição final; posição inicial; velocidade inicial; velocidade final e aceleração, portanto na faixa em que a posição final é 1 metro, temos que a diferença entre seus respectivos tempos médio está na casa dos milésimos de segundo, sendo contemplado com a variação do tempo médio com isso, dentro da “Margem de Erro”. E com a simples observação da tabela 3 é possível identificar esse mesmo comportamento nas outras faixas de medição.


\begin{table}[h!]
\caption{Resultados das medidas do MRUV}
    \centering
    \begin{tabular}{lll}
  \toprule
          Massa = 179,45 g\\
  x (cm) & \overline{t} (s) & \overline{t}^2 (s^2)\\
  \midrule
  40,00 & 1,620 & 2,624\\
  50,00 & 1,988 & 3,952\\
  60,00 & 2,319 & 5,378\\
  70,00 & 2,604 & 6,781\\
  80,00 & 2,841 & 8,071\\
  90,00 & 3,073 & 9,443\\
  100,00 & 3,299 & 10,883\\
  110,00 & 3,501 & 12,257\\
  120,00 & 3,698 & 13,675\\
  130,00 & 3,880 & 15,054\\
  \bottomrule
\end{tabular}
\label{tabela}
\legend{Fonte: Procedimento realizado em laboratório pelos integrantes}
\end{table}

\begin{table}[h!]
\caption{Resultados das medidas do MRUV}
    \centering
    \begin{tabular}{lll}
  \toprule
          Massa = 279,50 g\\
  x (cm) & \overline{t} (s) & \overline{t}^2 (s^2)\\
  \midrule
  40,00 & 1,616 & 2,611\\
  50,00 & 1,994 & 3,975\\
  60,00 & 2,302 & 5,301\\
  70,00 & 2,594 & 6,727\\
  80,00 & 2,847 & 8,103\\
  90,00 & 3,072 & 9,438\\
  100,00 & 3,302 & 10,903\\
  110,00 & 3,499 & 12,240\\
  120,00 & 3,696 & 13,659\\
  130,00 & 3,881 & 15,061\\
  \bottomrule
\end{tabular}
\label{tabela}
\legend{Fonte: Procedimento realizado em laboratório pelos integrantes}
\end{table}