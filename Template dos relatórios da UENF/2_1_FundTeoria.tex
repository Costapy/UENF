%Coloque a fundamentação teoria aqui. Coloque as principais equações. Veja o exemplo  %exemplo de como colocar duas citações ou mais.

%Veja o ambiente que a equação está colocada. 
% Apague e coloque o seu texto.
Para realizarmos esse experimento foi de grande valia a fundamentação teórica obtida na disciplina de Física 1, de posse dessas informações conseguimos identificar melhor o tipo do movimento e as forças que atuam no corpo em questão além de medirmos o tempo para que o evento se concretizasse. 

Como esse experimento requer uma aceleração constante, o ideal é usarmos a natureza a nosso favor e utilizarmos uma constante física para fornecer essa aceleração constante que precisamos, nada melhor do que a gravidade para nos auxiliar nesse experimento, entretanto a gravidade atrai tudo para o centro da terra, todavia nosso material utilizado para esse experimento é de características retilíneas, portanto devemos utilizar apenas uma porcentagem da gravidade de modo que possamos conciliar o nosso instrumento com a nossa grandeza constante. Nesse instante deve entrar em ação o nosso conhecimento sobre vetores, ora se o vetor gravidade atrai tudo para o centro da terra, na direção y, o melhor caminho é decompor esse vetor utilizando um ângulo entre a superfície do trilho de ar e o próprio trilho de ar. Decompondo esse vetor gravidade teremos que a componente x desse vetor vale o módulo do vetor gravidade multiplicado pelo cosseno do ângulo entre a superfície e o trilho de ar. Logo esse será o nosso valor referência para aceleração, tendo em vista que a gravidade é constante e que esse ângulo também é constante.

Fórmulas usadas na construção do relatório:

\begin{equation}\label{eq1}
\sigma = \sqrt{\frac{\sum_{i=1}^n (t_i-\overline{t})^2}{(n-1)}}
\end{equation}

\begin{equation}\label{eq2}
\Delta \overline{t}=\frac{\sigma}{\sqrt n}
\end{equation}

\begin{equation}\label{eq2}
x=x_0+v_0t+\frac{at^2}{2}
\end{equation}
